\documentclass[11pt]{article}

% Page geometry
\usepackage{geometry}
\geometry{
  margin=0.5in,% Equal margin of 0.5in from all four sides
  includefoot% Include only the footer in the margin calculations, since you don't have a header
}

% Header/footer
\usepackage{fancyhdr}
\pagestyle{fancy}% Page style will be fancy
\fancyhf{}% Clear headers and footers
\renewcommand{\headrulewidth}{0pt}% No header rule
%\renewcommand{\footrulewidth}{0pt}% No footer rule (default)
\fancyfoot[L]{Ryan D. Mueller} %\quad %\href{mailto:kathryn.haglin@appc.upenn.edu}{{\small kathryn.haglin@appc.upenn.edu}}}
\fancyfoot[C]{\textit{Cover Letter}}
\fancyfoot[R]{\thepage}

\usepackage{hyperref,graphicx}

\usepackage{color}
\setlength{\parindent}{0pt}% No paragraph indentation
\setlength{\parskip}{.5\baselineskip plus 0.1\baselineskip minus 0.1\baselineskip}

\begin{document}

%\includegraphics[width=2in]{TAM-PrimaryMarkA.png}% Logo
\hfill% FROM details
\begin{tabular}{@{}r@{}}
  \today \\
  [.5em]
Texas A\&M University \\
Physics and Astronomy Dept. \\
4242 TAMU \\
College Station, TX 77843-4242 \\
[.5em]
(281) 415-5146 \\
\href{mailto:rmueller@physics.tamu.edu}{rmueller@physics.tamu.edu}\\
%\href{http://khaglin.weebly.com/}{http://khaglin.weebly.com/}
\end{tabular}



\bigskip

% TO details
%\begin{tabular}{@{}l@{}}
%  Mrs.\ Jane Smith \\
%  Recruitment Officer \\
%  The Corporation \\
%  123 Pleasant Lane \\
%  City, State 12345
%\end{tabular}

% Rochester ------------------------------------------------
%\begin{tabular}{@{}l@{}}
% Department of Political Science \\
% University of Rochester \\
% Harkness Hall 333 \\
% Rochester, New York 14627-0146
%\end{tabular}
% -----------------------------------------------------------------------

% Emory ------------------------------------------------
%\begin{tabular}{@{}l@{}}
% Department of Political Science \\
% Emory University \\
% 327 Tarbutton Hall \\
% 1555 Dickey Drive \\
% Atlanta, GA 30322
%\end{tabular}
% -----------------------------------------------------------------------

% Brown ------------------------------------------------
%\begin{tabular}{@{}l@{}}
% Department of Political Science \\
% Brown University \\
% 36 Prospect Street \\
% Providence, RI 02912
%\end{tabular}
% -----------------------------------------------------------------------

% MIT ------------------------------------------------
%\begin{tabular}{@{}l@{}}
%Department of Political Science \\
% MIT \\
% E53-470 \\
% 77 Massachusetts Avenue \\
% Cambridge, MA 02139
%\end{tabular}
% -----------------------------------------------------------------------

% UMN ------------------------------------------------
\begin{tabular}{@{}l@{}}
School of Physics \& Astronomy \\
 University of Minnesota \\
116 Church Street S.E. \\
Minneapolis MN, 55455
\end{tabular}
% -----------------------------------------------------------------------


% McGill ------------------------------------------------
%\begin{tabular}{@{}l@{}}
%Department of Political Science \\
% McGill University \\
% Room 414, Leacock Building \\
%855 Sherbrooke Street West \\
%Montreal, Quebec H3A 2T7 \\
%\end{tabular}
% -----------------------------------------------------------------------

% UNC ------------------------------------------------
%\begin{tabular}{@{}l@{}}
%Department of Political Science \\
%The University of North Carolina at Chapel Hill \\
%361 Hamilton Hall, CB 3265 \\  
%Chapel Hill, NC 27599-326
%\end{tabular}
% -----------------------------------------------------------------------

% USC ------------------------------------------------
%\begin{tabular}{@{}l@{}}
%Department of Political Science \\
%University of Southern California \\ 
%3518 Trousdale Parkway \\
%Von Kleinsmid Center (VKC), Room 327 \\
%Los Angeles, CA 90089-0044
%\end{tabular}
% -----------------------------------------------------------------------

% UC Irvine ------------------------------------------------
%\begin{tabular}{@{}l@{}}
%Department of Political Science \\
%University of California, Irvine \\ 
%3151 Social Science Plaza \\
%Irvine, CA 92697-5100
%\end{tabular}
% -----------------------------------------------------------------------

% Boulder ------------------------------------------------
%\begin{tabular}{@{}l@{}}
%Department of Political Science \\
%University of Colorado, Boulder \\
%333 UCB \\
%Boulder, CO 80309-0333
%\end{tabular}
% -----------------------------------------------------------------------
% MSU ------------------------------------------------
%\begin{tabular}{@{}l@{}}
%Department of Political Science \\
%%Michigan State University \\
 %303 South Kedzie Hall \\
%East Lansing, MI 48824
%\end{tabular}
% -----------------------------------------------------------------------

% TCU ------------------------------------------------
%\begin{tabular}{@{}l@{}}
%Department of Political Science \\
%Texas Christian University \\
%2855 Main Drive \\
%Scharbauer Hall 2007 \\
%Fort Worth, TX 76129
%\end{tabular}
% -----------------------------------------------------------------------


% URI ------------------------------------------------
%\begin{tabular}{@{}l@{}}
%Department of Political Science \\
%University of Rhode Island \\
%206 Washburn Hall \\
% 80 Upper College Road \\
%  Kingston, RI 02881
%\end{tabular}
% -----------------------------------------------------------------------

% Purdue ------------------------------------------------
%\begin{tabular}{@{}l@{}}
%College of Liberal Arts \\
%Purdue University \\
%100 N.\ University St\\
%Beering Hall\\
%West Lafayette, IN 47907
%\end{tabular}
% -----------------------------------------------------------------------

% Penn State -------------------------------
%\begin{tabular}{@{}l@{}}
%Department of Political Science \\
%Penn State University \\
%203 Pond Lab \\
%University Park, PA 16802\\
%\end{tabular}
%  --------------------------------

% Emory -------------------------------
%\begin{tabular}{@{}l@{}}
%%Department of Political Science \\
%Emory University \\
%327 Tarbutton Hall \\
%1555 Dickey Drive \\
%Atlanta, GA 30322
%\end{tabular}
%  --------------------------------

% Georgia -------------------------------
%\begin{tabular}{@{}l@{}}
%Department of Political Science \\
%School of Public and International Affairs \\
%The University of Georgia \\
%302 Baldwin Hall \\
%Athens, GA 30602 \\
%\end{tabular}
%  --------------------------------

\bigskip

To the Search Committee: 

I am a CMS PhD student at Texas A\&M University searching for novel particles and working on track based muon alignment. I am writing to apply for the CMS/FAIR4HEP postdoctoral position offered by the University of Minnesota. At CMS, I have developed technical, phenomenological, leadership skills, and I have expertise in muons and the muon systems. Through personal and professional interests, I have intensively studied both the theory and industry application of machine learning models, such as DNNs, transformers, etc... and made use of them in my personal work, such as my game theory inspired genetic algorithm designed to pick optimal brackets for March Madness pools. I am also famous for my \href{https://www.symmetrymagazine.org/article/the-sourdough-starter-physics-family}{bread} at the Lhc Physics Center. 

%I believe I am an ideal candidate for this position as it both meshes with my professional development goals, and my personal expertise's will allow me to contribute quickly to the FAIR4HEP project and Higgs CMS analysis. 

Open and accessible data is a huge step towards moving CMS analysis, and HEP goals in general, forward, and I am thrilled at the chance to work on the FAIR4HEP program. My experience in CMS and with the machine learning and data science communities will allow me to approach goals in the FAIR4HEP project both as a physicists and a data scientist. My in depth experience working with physics analysis, phenomenology projects, muon reconstruction, top tagging and more means I understand the needs of CMS and the HEP community implicitly, and can bring my knowledge to the table to ensure the tools developed are prioritized and meaningfully contribute to areas of need. My knowledge of the practice, theory, and history of machine learning and data science means I understand the requirements of FAIR CMS data, how it will need to be shared with the machine learning communities, what tools will need to be developed, and how to communicate all this with subject matter experts. The sort of breakout successes of AlexNet/ImageNet, or Alpha Fold 2, at CMS will require knowledge of both domains I believe I provide that. Beyond breakout improvements in object reconstruction, NN used in FPGAs for triggers, or physics analysis, I believe with the right vision, the developments from FAIR CMS data can help bring modern data analyses tools to CMS. This can, for example, provide "quality of life" improvements to the tools physicists use, enabling them to focus more on physics and less on technical issues. 

%FAIR CMS data will not only help create new tools allowing physicists to focus more on the physics while achieving better results, but it will also increase the footprint of CMS in the wider scientific and technical communities. The discoveries and tools created from this can have fair reaching impacts. I strongly believe in this mission, and I would love to help bring this vision about. 

%In addition, I have a strong interest and background in machine learning and AI, and I was intrigued by the mention of ImageNet in FAIR4HEP literature and in the job posting. The explosion of deep learning has both been driven by and mirrored by the impressive reduction in error rate in each years ImagNet Challenges, headlined by the famous breakout "AlexNet". Similar breakout improvements in quick decision making in FPGAs, object identification and tagging, or event selection and interpretation will require proper leadership and collaboration between the CMS community and ML/AI community. 

%My background in CMS physics analysis, software development, and technical work has given me an understanding of the computing and technical needs of modern high energy physics.

%For example, my dissertation focuses on the search for a novel Z' candidate through “Bottom Fermion Fusion”. In this work, I've coordinated and collaborated with intuitions across the CMS collaboration, developed analysis software, managed POG recommendations, optimized event selection and analysis strategy, produced skims, figures of merit and all other tasks needed to complete a CMS analysis. Beyond my dissertation, I've had the opportunity to work on a phenomenology project, developing a final state agnostic method to measure top quark chirality where I learned all about the grownig world of deep learning/AI for top tagging. In my service work in the Track Based Muon Alignment team, I've worked with lower level issues such as muon/object reconstruction, managing conditions and so on. 

%I understand the needs of HEP physicists, the nature of the data we work with, and what the next generation of tools will optimally do. On the flip side, my background in machine learning as a matter of serious self study allows me to also understand the challenges from the ML/AI communities side. I understand what needs to go into a FAIR HEP data set, and what non-physicist computer/data scientists and programmers will need and want to do their jobs. I can contribute on both sides, and more importantly, work with both sides. 

I am also excited to continue my physics research in CMS. My research experience at CMS includes in depth work on physics analyses, phenomenology studies, service work, and leadership to graduate and undergraduate students. At the University of Minnesota, I can lead the next generation Higgs analysis from start to finish and advise new graduate students. My experience helping guide younger students working on other Z' analyses and teaching new additions to the track based alignment team has provided me plenty of practice in leading students. I will continue pushing to advance my analysis, technical, leadership, and physics skills at my next position. With a new era of data taking approaching, I am eager to apply my physics and analysis skills to contribute to the next Higgs analyses at the University of Minnesota, and I am eager to lead a new generation of graduate students. I am also strongly motivated to expand my leadership skills and responsibilities within the CMS organization, and I would love the chance to take on a leadership role of a DPG, POG or PAG group.

%My experience, for example, as a member of the Track Based Muon Alignment team, has prepared me well to lead graduate students. I've had the opportunity to build relationships within CMS, guide and educate fellow graduate students and undergraduate students through complex projects and work with the technical aspects of the CMS muons system. To focus on my mentorship in the last year, I have worked extensively with two graduate students and one undergraduate student, both teaching them the skills they would need, and guiding them to the successful completion of tasks. Hands on leadership and mentorship has been crucial for training the next generation of grad students and accomplishing tasks effectively. I have also worked as the face of the Muon Alignment group, provided critical updates at AlcaDB, and helping lead inter-group cooperation and studies between the muon alignment and tracker alignment team. My experience here is perfect preparation to help guide graduate students and represent the University of Minnesota as a postdoc in the CMS organization. 



Lastly, during my stint as the LPC Distinguished Graduate Student at Fermilab, I was able to build connections within the USCMS community, both professionally and personally. For example, I both served as co-chair the LPC Computing Discussion group and helped start an unofficial LPC sourdough baking ring. Since then, I've kept active in the USCMS community, continuing to serve as a co-chair on the LPC Computing Discussion group and stayed active in other LPC activities. My connection with USCMS will be a valuable addition in my next position as a post doc. 


%My interest in the University of Minnesota also goes beyond my professional interests. My partner lives in Duluth and is a faculty member at the University of Minnesota-Duluth in the College of Liberal Arts. I would be excited to continue my academic journey in Minnesota and join the broader University of Minnesota community as well. 



 Please do not hesitate to contact me at rmueller@physics.tamu.edu, or at (281)-415-5146 should you have any questions. Thank you for your consideration, and I look forward to hearing from you soon.


\bigskip

\begin{tabular}{@{}l@{}}
Sincerely, \\
  [.4em]
% \includegraphics[width=1.5in]{signature}\\ % my signature
  %[.2em]
  Ryan D. Mueller
\end{tabular}
\end{document}




