\documentclass[11pt]{article}

% Page geometry
\usepackage{geometry}
\geometry{
  margin=0.5in,% Equal margin of 0.5in from all four sides
  includefoot% Include only the footer in the margin calculations, since you don't have a header
}

% Header/footer
\usepackage{fancyhdr}
\pagestyle{fancy}% Page style will be fancy
\fancyhf{}% Clear headers and footers
\renewcommand{\headrulewidth}{0pt}% No header rule
%\renewcommand{\footrulewidth}{0pt}% No footer rule (default)
\fancyfoot[L]{Ryan D. Mueller} %\quad %\href{mailto:kathryn.haglin@appc.upenn.edu}{{\small kathryn.haglin@appc.upenn.edu}}}
\fancyfoot[C]{\textit{Cover Letter}}
\fancyfoot[R]{\thepage}

\usepackage{subfig}
\usepackage{hyperref,graphicx}
\usepackage{color}

\hypersetup{
  linkcolor  = blue,
  citecolor  = blue,
  urlcolor   = blue,
  colorlinks = true,
}

\setlength{\parindent}{0pt}% No paragraph indentation
\setlength{\parskip}{.5\baselineskip plus 0.1\baselineskip minus 0.1\baselineskip}

\usepackage[
backend=biber,
style=alphabetic,
sorting=ynt
]{biblatex}

\addbibresource{mybibliography.bib}



\begin{document}

%\includegraphics[width=2in]{TAM-PrimaryMarkA.png}% Logo
\hfill% FROM details
\begin{tabular}{@{}r@{}}
  \today \\
  [.5em]
Texas A\&M University \\
Physics and Astronomy Dept. \\
4242 TAMU \\
College Station, TX 77843-4242 \\
[.5em]
(281) 415-5146 \\
\href{mailto:rmueller@physics.tamu.edu}{rmueller@physics.tamu.edu}\\
%\href{http://khaglin.weebly.com/}{http://khaglin.weebly.com/}
\end{tabular}



\bigskip


% UMN ------------------------------------------------
\begin{tabular}{@{}l@{}}
School of Physics \& Astronomy \\
 University of Minnesota \\
116 Church Street S.E. \\
Minneapolis MN, 55455
\end{tabular}
% -----------------------------------------------------------------------



\bigskip

My research experience at CMS includes in depth work on physics analyses, phenomenology studies, service work, and leadership to graduate and undergraduate students. I will continue pushing to advance my analysis, technical, leadership, and physics skills testing the standard model (e.g., search for for new, exotic particles) at my next position. I believe I would be a valuable member of the University of Minnesota team. 

My skill sets, interests, and goals all fit well with the postdoc opening at the University of Minnesota. My motivation at CMS is the search for new physics, and at my next position I hope to take a leadership role in new analyses with the SUSY group. 

I also have had a strong interested in machine learning, and have invested time over the last year studying up on state of the art theory and practice in deep neural networks and applying them in hobby projects. The chance to apply my interests and expertise in machine learning with my CMS work is an exciting opportunity. 

Working on the readout electronics for the Endcap Calorimeter would be a welcome diversification of my service work at CMS. I have circuit design experience, and I would love to contribute on the hardware side of service work.

Finally, I am strongly motivated to expand my leadership skills and responsibilities within the CMS organization, and I would love the chance to take on a leadership role of a DPG, POG or PAG group.

In the next few pages, I've highlighted some aspects of my current research. 

\section{BFF Z'}

My dissertation focuses on the search for a novel Z' candidate motivated by the recent discrepancy between standard model predicted and observed branching fractions of leptonic B meson decays to muons versus electrons noted at the lHCb ($R_K$ and $R_{K^{*}}$). The anomaly in $R_K$ and $R_{K^{*}}$ ,combined with Belle's measurements, nears 4$\sigma$\cite{rkStar}. This hint of lepton non-universality can be explained by the addition of a Z' particle which features a flavour changing $\delta_{bs}$ interaction, no coupling with lighter quarks, and no decay to first generation leptons. Because of it's unique couplings, we can search for the particle in conjunction with associated b-jets and probe lower mass points than other Z' analyses. 

Our search strategy is based on the strategy laid out in Abdullah et al. \cite{PhysRevD.97.075035}, tuned and re-optimized for 2016-2018 data from the CMS experiment. Possible production mechanisms of our Z' candidate include a standard s-channel process, as well as one and two associated b-jet cases (Figure \ref{fig:z_prime_cases}). The first order s-channel process in our model is covered in the inclusive Z' search, and a specialized search will not exclude extra phase space (see Figure \ref{fig:exc_pahse_space}). However, Z' production in association with an ISR b-jet provides us with an additional handle to observe low mass Z's with greater sensitivity. The second shown final state in Figure \ref{fig:z_prime_cases} is dubbed "Bottom Fermion Fusion", in analogy to Vector Boson Fusion. 


\begin{figure}[h]
    \centering
    \includegraphics[width=\textwidth]{images/BFF_z_prime_cases.png}
    \caption{Z' production topologies at a hadron collider.}
    \label{fig:z_prime_cases}
\end{figure}


%\begin{figure}[h]
%    \centering
%    $n_{SR}=n_{CR_{b}^{ee}}\frac{n_{CR_{no-b}^{\mu\mu}}}{n_{CR_{no-b}^{ee}}}$
%    \caption{The basic equation for our data-driven background estimation. The ratio of the muon-anti-b region to the electron-anti-b region provides a scale factor, converting between electrons and muons. This multiplied by the count in the electron+}
    \label{fig:ABCD_eqn}
%\end{figure}


\begin{figure}
    \centering
    \includegraphics[width=8cm]{images/excluded_phase_space.png}
    \caption{Reach of b-associated Z' search at the LHC compared to the inclusive search. The top right grey area represents constraints from $B_s\rightarrow \bar{B}_s$ mixing, the bottom left represents constraints from neutrino trident production/g-2, and the green and yellow bands represent 1 and 2 standard deviation bands from the B anomalies. We see that the b-associated Z' search (red) excludes more phase space than the inclusive search (green).}
    \label{fig:exc_pahse_space}
\end{figure}


My dissertation focuses on the muonic decay of the Z'. Selecting for events containing dilepton ($e$ and $\mu$) pairs, low missing $E_T$, and at least one jet, we divide out sample into one and two jet, b-tagged and anti-btagged, and lepton permutation ($ee$, $e\mu$, $\mu\mu$) bins. We then employ three kinematic cuts to increase our significance: A $MET/M_{\ell\ell}$ (aka relative MET) filter takes advantage of the low transverse MET and high mass lepton pairs of the signal. Filtering on the difference between the hadronic transverse energy and letponic transverse energy (HT-LT) takes advantage of the fact that our associated b-jets tend to be "soft", or low energy. Finally, we permute the b-jets and leptons to create a best-guess top quark decay and compute it's mass (Top Mass Bound) to exclude top quarks from the final events. Cut values on these variables are optimized in monte-carlo simulation, and HT-LT and relative met cuts are optimized for variable mass points. Finally, an ABCD method is used to create a data driven background estimation, where we use electron vs muon and b-tagged vs anti-b-tagged regions to define our control and background regions. 

We are currently finalizing our statistical analysis and systematic uncertainties on the analysis. 

\section{Muon Alignment}

 Track based muon alignment is a critical alignment and calibration task at CMS. The alignment team is responsible for delivering up-to-date conditions of the positions of the muon detectors, ensuring the muon system is operating optimally, and developing further improvements to the muon alignment algorithm. I have contributed to all aspects of track based muon alignment since joining CMS in 2014, from delivering conditions under tight deadlines to ensuring proper data integrity, leading group and inter-group investigations, teaching and guiding new members of muon alignment, and developing new tools for the track based muon alignment group. 
 
 Some examples of my contributions include working with other members of the team to upgrade muon alignment in the DTs from 3 to 6 degrees of freedom, upgrading our alignment validation code to function modularly and significantly more efficiently, leading multiple investigations into unexpected features in alignment, leading a study to measure overall systematic and statistical uncertainties, creating a simplified standalone monte-carlo simulation of the muon alignment algorithm for educational purposes and for directed investigations into hard to investigate systematics, branching the toy mc simulation of muon alignment into a project for a new student, helping guide development of GEM alignment, upgrading alignment software, representing the Muon Alignment group at AlcaDB/DPG meetings and much more. Our work in recent years is currently being summarized in a new Muon Alignment note. I will bring my leadership and innovation demonstrated by my work in muon alignment to future research. 
 
 
 \section{MonoTop}
 
 Exotic particle decays (e.g. certain non-thermal dark matter models) can result in single top production with right chirality at the LHC. Since standard model single top production results in mostly left-chiral tops, if we can measure the chirality of the top quark, we can gain a better handle on these processes. However, traditional top chirality measurements are designed to work with $t\bar{t}$ events and leverage correlations between the top quark decay products to measure chirality. e.g. the angle between the lepton angles of both pair-produced top quarks. At A\&M, we developed a simple single top observable to measure chirality in an event agnostic manner.
 
Our measurable relies on the relative energy difference a single top quark's daughter particles, allowing it to function independently of the top quark production mechanism. This not only makes it useful for our monotop search, but also as a general technique that could be applied to many top based analysis (e.g. precision standard model measurements). 

\section{Trajectory}


I am excited about the potential to work in an up-and-coming group at the University of Minnesota with Nadja Strobe, conducting new physics searches, developing new machine learning models, getting my hands dirty with the Calorimter readout electronics, and stepping up as a leader at the University of Minnesota and in CMS as a whole.



\bigskip

\begin{tabular}{@{}l@{}}
Sincerely, \\
  [.4em]
% \includegraphics[width=1.5in]{signature}\\ % my signature
  %[.2em]
  Ryan D. Mueller
\end{tabular}


\medskip

\printbibliography

\end{document}



%R. Allahverdi, M. Dalchenko, B. Dutta, A. Florez, Y. Gao, T. Kamon, N. Kolev, \textbf{R. Mueller}, and M. Segura. \href{https://link.springer.com/article/10.1007/JHEP12(2016)046}{``Distinguishing standard model extensions using monotop chirality at the LHC."} J. High Energ. Phys. (2016) 2016: 46. doi:10.1007/JHEP12(2016)046

